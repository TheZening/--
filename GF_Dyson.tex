%==============================
%==============================
\section{格林函数: Dyson方程}
\label{Sec: Dysong方程}


我们之前的学习, 都是在一个宏观层次上讨论格林函数: 对于一个不含时的 Hamiltonian, 我们从来没有问过他的本征态本征值到底怎么算出来的, 而是直接给出:
\begin{equation}
\hat{H} \ket{n} = E_n \ket{n}
\end{equation}
然后我们可以写出他的时域推迟格林函数:
\begin{equation}
\hat{G}^R(t-t') = \frac{1}{\mathrm{i}\,\hbar} \theta(t-t') \mathrm{e}^{-\mathrm{i}\,\hat{H}(t-t')/\hbar}
\end{equation}
以及频域推迟格林函数:
\begin{equation}
\hat{G}^R(E) = \lim_{\eta \to 0^+} \frac{1}{E - \hat{H} + \mathrm{i}\,\eta}
\end{equation}
其中我们不再给频域的格林函数标上 $\tilde{}$ 符号, 要大家注意区分一下了.


可是我们从来没有问过, 这个 Hamiltonian 究竟是怎么解的?
根据量子力学的学习, 我们知道, 只有非常有限的几个模型, 我们才能够解析地求出 Hamiltonian 的本征态和本征值.
对于大部分的 Hamiltonian, 我们只能通过扰动计算, 数值计算等方法, 来近似地求出本征态和本征值.


仿照这个思路, 我们现在看看我们能否仍然把 Hamiltonian 分成两部分:
\begin{equation}
\hat{H} = \hat{H}_0 + \hat{V}
\end{equation}
其中 $\hat{H}_0$ 是一个我们能够解析地求出本征态和本征值的 Hamiltonian, 比如自由粒子, 简谐振子, 无限深势阱等.
而 $\hat{V}$ 则是一个我们无法解析地求解的势能干扰, 比如一个随机分布的杂质势, 一个复杂的外场等等, 总而言之就是, 一旦有了 $\hat{V}$, 我们就无法解析地求出 $\hat{H}$ 的本征态和本征值了.

\question{那么, 我们能否通过 $\hat{H}_0$ 的格林函数, 配合一些计算方法, 来扰动的求解出来 $\hat{H}$ 的动力学响应啥的呢?}
这就是 \newterm{Dyson 方程}要解决的问题.
为了和 QFT 里面的 Dyson 方程区分开来, 我们把这个 Dyson 方程叫做 ``单粒子 量子力学 Dyson 方程''.
\reminder{而且, 我们这里暂时不假设 $\hat{V}$ 很小, 也就是说, 我们这个扰动计算, 并不是小扰动计算, 这个边学边体会即可}.


现在我们管所有能求解的 Hamiltonian 的部分 $\hat{H}_0$ 称为自由 Hamiltonian, 而把无法求解的部分 $\hat{V}$ 称为相互作用.
这样, 对于 $\hat{H}_0$, 我们可以写出他的频域推迟格林函数:
\begin{equation}
\hat{G}_0^R(E) = \lim_{\eta \to 0^+} \frac{1}{E - \hat{H}_0 + \mathrm{i}\,\eta}
\end{equation}
这个下标 $0$ 就表示这是自由粒子的格林函数.
而对于完整的 Hamiltonian $\hat{H}$, 我们同样可以写出他的频域推迟格林函数:
\begin{equation}
\hat{G}^R(E) = \lim_{\eta \to 0^+} \frac{1}{E - \hat{H} + \mathrm{i}\,\eta}
\end{equation}
现在我们要做的, 就是通过 $\hat{G}_0^R(E)$ 来求解 $\hat{G}^R(E)$.


\begin{mtheorem}{一个代数恒等式}
对于任意两个算符 $\hat{A}$ 和 $\hat{B}$, 如果他们是非奇异的 (即可逆的) 算符, 那么下面的代数恒等式成立:
\begin{equation}
\hat{A}^{-1} - \hat{B}^{-1} = \hat{B}^{-1} (\hat{B} - \hat{A}) \hat{A}^{-1}
\end{equation}
形式上, 写成分母的形式就是:
\begin{equation}
\frac{1}{\hat{A}} - \frac{1}{\hat{B}} = \frac{1}{\hat{B}} (\hat{B} - \hat{A}) \frac{1}{\hat{A}}
\end{equation}
\end{mtheorem}
这个证明几乎是一眼就能看出来, 不多说了.


现在我们把上面的定理应用到 $\hat{A} = E - \hat{H} + \mathrm{i}\,\eta$ 和 $\hat{B} = E - \hat{H}_0 + \mathrm{i}\,\eta$ 上, 那么就有:
\begin{equation}
\begin{aligned}
\hat{G}^R(E) - \hat{G}_0^R(E)
&= \hat{G}_0^R(E) \big[ (E - \hat{H}_0 + \mathrm{i}\,\eta) - (E - \hat{H} + \mathrm{i}\,\eta) \big] \hat{G}^R(E) \\
&= \hat{G}_0^R(E) (\hat{H} - \hat{H}_0) \hat{G}^R(E) \\
&= \hat{G}_0^R(E) \hat{V} \hat{G}^R(E)
\end{aligned}
\end{equation}
也就是说, 我们得到了 Dyson 方程:
\begin{mdefinition}{Dyson 方程}
\begin{equation}
\hat{G}^R(E) = \hat{G}_0^R(E) + \hat{G}_0^R(E) \hat{V} \hat{G}^R(E)
\end{equation}
\end{mdefinition}


在代数意义上求解这个方程, 我们可以把他变形为:
\begin{equation}
\hat{G}^R(E) = \frac{\hat{G}_0^R(E)}{1 - \hat{V} \hat{G}_0^R(E)} = \frac{1}{(\hat{G}_0^R(E))^{-1} - \hat{V}} = \frac{1}{E - \hat{H}_0 - \hat{V} + \mathrm{i}\,\eta} = \frac{1}{E - \hat{H} + \mathrm{i}\,\eta}
\end{equation}
此处, 我们没有对 $V$ 做任何近似, 只是单纯地把 Dyson 方程变形了一下, \reminder{这个式子是能够处理强耦合情况的}: 奇点被反应在分母里面了.


我们根据几何级数的学习, 我们知道, 如果 $\hat{V} \hat{G}_0^R(E)$ 的谱半径小于 $1$, 那么我们可以把上面的式子展开成:
\begin{equation}
\hat{G}^R(E) = \hat{G}_0^R(E) + \hat{G}_0^R(E) \hat{V} \hat{G}_0^R(E) + \hat{G}_0^R(E) \hat{V} \hat{G}_0^R(E) \hat{V} \hat{G}_0^R(E) + \cdots
\end{equation}
这就是 \newterm{Born 级数}.
频域的图像数令人困惑的, 我们现在把他变到时域上来看看:
\begin{equation}
\hat{G}^R(\tau) = \frac{1}{2\pi\hbar} \lim_{\eta \to 0^+} \int_{-\infty}^{+\infty} \hat{G}^R(E) \mathrm{e}^{-\mathrm{i}\,E \tau/\hbar} \dd{E}
\end{equation}
注意, 我们这前面的系数是 $1/(2\pi\hbar)$, 因为我们定义的频域格林函数的变量是能量 $E$, 而不是角频率 $\omega$, 这样不用担心单位对不上号.
计算这个积分, 我们需要使用 Cauchy 积分公式, 这里就不赘述了, 直接给出结果:
\begin{equation}
\hat{G}^R(\tau) = \frac{1}{\mathrm{i}\,\hbar} \theta(\tau) \mathrm{e}^{-\mathrm{i}\,\hat{H} \tau/\hbar}
\end{equation}
就是我们之前算的时域推迟格林函数.
同样地, 我们也可以把 $\hat{G}_0^R(E)$ 变到时域上:
\begin{equation}
\hat{G}_0^R(\tau) = \frac{1}{\mathrm{i}\,\hbar} \theta(\tau) \mathrm{e}^{-\mathrm{i}\,\hat{H}_0 \tau/\hbar}
\end{equation}
现在我们把 Born 级数也变到时域上来, 这里面有一个小技巧可以用: 因为频域的乘积在时域上是卷积!
考虑到:
\begin{equation}
\frac{1}{2\pi\hbar}\int_{-\infty}^{+\infty} \hat{G}_0^R(E) \hat{V} \hat{G}_0^R(E) \mathrm{e}^{-\mathrm{i}\,E \tau/\hbar} \dd{E}
\end{equation}
把其中一个 $\hat{G}_0^R(E)$ 变到时域上:
\begin{equation}
= \frac{1}{2\pi\hbar}\int_{-\infty}^{+\infty} \hat{G}_0^R(\tau_1) \mathrm{e}^{\mathrm{i}\,E \tau_1/\hbar} \dd{\tau_1} \hat{V} \hat{G}_0^R(E) \mathrm{e}^{-\mathrm{i}\,E \tau/\hbar} \dd{E}
\end{equation}
交换积分顺序:
\begin{equation}
= \int_{-\infty}^{+\infty} \hat{G}_0^R(\tau_1) \hat{V} \left[ \frac{1}{2\pi\hbar} \int_{-\infty}^{+\infty} \hat{G}_0^R(E) \mathrm{e}^{-\mathrm{i}\,E (\tau - \tau_1)/\hbar} \dd{E} \right] \dd{\tau_1}
\end{equation}
注意到中括号里面的就是 $\hat{G}_0^R(\tau - \tau_1)$, 所以我们得到:
\begin{equation}
= \int_{-\infty}^{+\infty} \hat{G}_0^R(\tau_1) \hat{V} \hat{G}_0^R(\tau - \tau_1) \dd{\tau_1}
\end{equation}
我们要是展开这个式子:
\begin{equation}
\int_{-\infty}^{+\infty} \frac{1}{\mathrm{i}\,\hbar} \theta(\tau_1) \mathrm{e}^{-\mathrm{i}\,\hat{H}_0 \tau_1/\hbar} \hat{V} \frac{1}{\mathrm{i}\,\hbar} \theta(\tau - \tau_1) \mathrm{e}^{-\mathrm{i}\,\hat{H}_0 (\tau - \tau_1)/\hbar} \dd{\tau_1}
\end{equation}
这样物理图像就清楚了: 当我们把这一项作用在一个态上, 首先这个态要经过时间 $\tau-\tau_1$ 的自由演化, 然后受到相互作用 $\hat{V}$ 的扰动, 最后再经过时间 $\tau - \tau_1$ 的自由演化.
注意到 $\theta(\tau_1) \theta(\tau-\tau_1)$ 的存在, 说明 $\tau_1$ 的积分区间其实是从 $0$ 到 $\tau$ 的.
中间这个积分变量 $\tau_1$ 描述了相互作用发生的时间点是 $\tau-\tau_1$, 对他做积分, 就把所有可能的相互作用时间点都考虑进去了.
\reminder{我们不知道粒子到底是在哪一瞬间撞到了, 根据量子力学的叠加原理, 我们必须把 `刚出门就撞了' 和 `走了一会儿才撞' 这些可能性都考虑进去, 这就是量子力学的本质所在}.


不需要繁复的推导, 我们只要用同样的思路, 就能写出 Born 级数的第三项:
\begin{equation}
\int_{0}^{\tau} \int_{0}^{\tau_1} \hat{G}_0^R(\tau_2) \hat{V} \hat{G}_0^R(\tau_1) \hat{V} \hat{G}_0^R(\tau - \tau_1 -\tau_2) \dd{\tau_2} \dd{\tau_1}
\end{equation}
啥意思呢, 就是, 粒子先经过时间 $\tau-\tau_1-\tau_2$ 的自由演化, 然后在时间点 $\tau-\tau_1-\tau_2$ 受到一次相互作用 $\hat{V}$ 的扰动, 然后经过时间 $\tau_1$ 的自由演化, 然后在时间点 $\tau-\tau_1$ 受到第二次相互作用 $\hat{V}$ 的扰动, 最后经过时间 $\tau_2$ 的自由演化.
积分自然就是把所有可能的相互作用时间点都考虑进去, 所以 $\tau_2$ 的积分区间是从 $0$ 到 $\tau_1$, 而 $\tau_1$ 的积分区间是从 $0$ 到 $\tau$.


综上所述, 我们可以把 Born 级数写成时域的形式:
\begin{equation}
\begin{aligned}
\hat{G}^R(\tau) =\; & \hat{G}_0^R(\tau) \\
& + \int_{0}^{\tau} \hat{G}_0^R(\tau_1) \hat{V} \hat{G}_0^R(\tau - \tau_1) \dd{\tau_1} \\
& + \int_{0}^{\tau} \int_{0}^{\tau_1} \hat{G}_0^R(\tau_2) \hat{V} \hat{G}_0^R(\tau_1) \hat{V} \hat{G}_0^R(\tau - \tau_1 -\tau_2) \dd{\tau_2} \dd{\tau_1} \\
& + \cdots
\end{aligned}
\end{equation}
这个式子看起来是不是更直观了一些?
这个式子告诉我们, 如果我们知道了自由粒子的格林函数 $\hat{G}_0^R(\tau)$, 那么我们就可以通过不断地把相互作用 $\hat{V}$ 插入到自由粒子的传播过程中, 来得到完整的格林函数 $\hat{G}^R(\tau)$.
这就是 Dyson 方程和 Born 级数的物理图像.


小小总结一下, Dyson 方程告诉我们:
\begin{equation}
\hat{G}^R(E) = \hat{G}_0^R(E) + \hat{G}_0^R(E) \hat{V} \hat{G}^R(E) 
\Rightarrow
\hat{G}^R(\tau) = \hat{G}_0^R(\tau) + \int_{-\infty}^{+\infty} \hat{G}_0^R(\tau_1) \hat{V} \hat{G}^R(\tau - \tau_1) \dd{\tau_1}
\end{equation}
这两个式子都说了同一个事情: 现在的格林函数等于没有相互作用时的格林函数, 加上过去所有时刻发生的相互作用对现在的格林函数的贡献.
如果需要非微扰求解, 我们就需要考虑 Dyson 方程的代数形式, 如果需要微扰求解, 那么我们就可以使用 Born 级数展开.


\comment{迭代求解也需要假设 $\hat{V}$ 很小吗?}
答案是肯定的, 因为迭代求解也需要收敛, 否则会得到一个没有意义的发散结果, 所以迭代求解同样需要假设 $\hat{V}$ 很小 (体现在余项很小):
\begin{equation}
\lim_{N \to \infty} (\hat{G}_0^R(E) \hat{V})^{N+1} \hat{G}^R(E) = 0
\end{equation}

\begin{mexample}{谐振子+扰动势能}
考虑一个一维谐振子, 他的 Hamiltonian 是:
\begin{equation}
\hat{H}_0 = \frac{\hat{p}^2}{2m} + \frac{1}{2} m \omega^2 \hat{x}^2
\end{equation}
他的本征态和本征值是众所周知的:
\begin{equation}
\hat{H}_0 \ket{n} = \hbar \omega \left( n + \frac{1}{2} \right) \ket{n}, \quad n = 0, 1, 2, \ldots
\end{equation}
现在我们给这个谐振子加上一个扰动势能:
\begin{equation}
\hat{V} = \lambda \delta(x)
\end{equation}
其中 $\lambda$ 是一个很小的常数.
现在我们想要求解完整的 Hamiltonian:
\begin{equation}
\hat{H} = \hat{H}_0 + \hat{V}
\end{equation}
的格林函数 $\hat{G}^R(E)$, 以及新的能级.
我们先写出自由谐振子的频域格林函数的矩阵元 (因为波函数我们都知道了, 直接写就好了):
\begin{equation}
G_0^R(x, x'; E) = \sum_{n=0}^{\infty} \frac{\psi_n(x) \psi_n^*(x')}{E - \hbar \omega (n + 1/2) + \mathrm{i}\,\eta}
\end{equation}
现在我们使用 Dyson 方程的矩阵元形式:
\begin{equation}
G^R(x, x'; E) = G_0^R(x, x'; E) + \int_{-\infty}^{+\infty} G_0^R(x, x''; E) V(x'') G^R(x'', x'; E) \dd{x''}
\end{equation}
把 $\hat{V}$ 代入进去:
\begin{equation}
G^R(x, x'; E) = G_0^R(x, x'; E) + \lambda G_0^R(x, 0; E) G^R(0, x'; E)
\end{equation}


现在我们先考虑精确求解: 非扰动求解的话, 我们把上面的式子在 $x=0$ 处取值:
\begin{equation}
G^R(0, x'; E) = G_0^R(0, x'; E) + \lambda G_0^R(0, 0; E) G^R(0, x'; E)
\end{equation}
解出 $G^R(0, x'; E)$:
\begin{equation}
G^R(0, x'; E) = \frac{G_0^R(0, x'; E)}{1 - \lambda G_0^R(0, 0; E)}
\end{equation}
把他代回去:
\begin{equation}
G^R(x, x'; E) = G_0^R(x, x'; E) + \frac{\lambda G_0^R(x, 0; E) G_0^R(0, x'; E)}{1 - \lambda G_0^R(0, 0; E)}
\end{equation}
新的能级就是这个格林函数的极点.
极点在哪里呢?
第一个可能性就是 $G_0^R(x, x'; E)$ 的极点, 也就是原来的能级.
对于奇宇称的能级, 由于波函数在 $x=0$ 处为零, 所以 $G_0^R(0, 0; E)$ 在这些能级处并不发散, 所以这些能级并不会被扰动势能影响.
而对于偶宇称的能级, 由于波函数在 $x=0$ 处不为零, 所以 $G_0^R(0, 0; E)$ 在这些能级处发散, 所以这些能级会被扰动势能影响.
第二个可能性就是分母为零:
\begin{equation}
1 - \lambda G_0^R(0, 0; E) = 0 \Rightarrow G_0^R(0, 0; E) = \frac{1}{\lambda}
\end{equation}
也就是:
\begin{equation}
\sum_{n= \text{even}} \frac{|\psi_n(0)|^2}{E - \hbar \omega (n + 1/2) + \mathrm{i}\,\eta} = \frac{1}{\lambda}
\end{equation}
求解这个方程, 就能得到被扰动势能影响的偶宇称能级的新位置, 能给出所有的新的能噗.
但是这里也体现出一个问题: 我们怎么真的把所有的波函数都加起来呢?
所以不可避免的, 还是会出现很多截断问题, 除非是有限维空间的问题, 否则我们还是没法完全求解出来.
不过好在, 通过这个式子, 我们已经把问题简化到一个只需要求解一个代数方程的问题了.
\end{mexample}


现在我们来看看怎么用格林函数, 尤其是微扰的情况下, 具体求出能级, 最好是能和我们之前学过的微扰理论联系起来.
我们从 Born 级数开始:
\begin{equation}
\hat{G}^R(E) = \hat{G}_0^R(E) + \hat{G}_0^R(E) \hat{V} \hat{G}_0^R(E) + \hat{G}_0^R(E) \hat{V} \hat{G}_0^R(E) \hat{V} \hat{G}_0^R(E) + \cdots
\end{equation}
如果只考虑第一阶:
\begin{equation}
\hat{G}^R(E) \approx \hat{G}_0^R(E) + \hat{G}_0^R(E) \hat{V} \hat{G}_0^R(E)
\end{equation}
有的文献也会说: 在 Dyson 方程中, 用 $\hat{G}_0^R(E)$ 代替 $\hat{G}^R(E)$ 来近似求解格林函数.
这个物理意义也很清楚: 我们只考虑粒子经历一次相互作用 $\hat{V}$ 的过程, 也就是所谓的 ``单次散射'' 过程.
我们现在暂时忽略掉推迟格林函数的 $\mathrm{i}\,\eta$ 项, 反正取了极限 $\eta \to 0^+$ 以后, 他只是起到一个规定极点绕过方式的作用, 不影响我们现在的讨论.
要是想求一个能级的微扰修正, 自然是把格林函数的矩阵元算出来:
\begin{equation}
\mel{n}{\hat{G}^R(E)}{n} = \mel{n}{\hat{G}_0^R(E)}{n} + \mel{n}{\hat{G}_0^R(E) \hat{V} \hat{G}_0^R(E)}{n}
\end{equation}
我们先算第一项:
\begin{equation}
\mel{n}{\hat{G}_0^R(E)}{n} = \frac{1}{E - E_n^{(0)}}
\end{equation}
其中 $E_n^{(0)}$ 是 $\hat{H}_0$ 的本征值.
我们看第二项哈, $G_0^R(E)$ 是 $\hat{H}_0$ 的函数, 所以他和 $\hat{H}_0$ 有相同的本征态:
\begin{equation}
\mel{n}{\hat{G}_0^R(E) \hat{V} \hat{G}_0^R(E)}{n} = \frac{\mel{n}{\hat{V}}{n}}{(E - E_n^{(0)})^2}
\end{equation}
所以我们得到:
\begin{equation}
\mel{n}{\hat{G}^R(E)}{n} = \frac{1}{E - E_n^{(0)}} + \frac{\mel{n}{\hat{V}}{n}}{(E-E_n^{(0)})^2} = \frac{1}{E - \left( E_n^{(0)} + \mel{n}{\hat{V}}{n} \right)}
\end{equation}
我们考虑到:
\begin{equation}
\frac{1}{x-a} = \frac{1}{x} + \frac{a}{x^2}
\end{equation}
所有后面两项可以加起来:
\begin{equation}
\frac{1}{E - E_n^{(0)}- \mel{n}{\hat{V}}{n}} = \frac{1}{E - E_n^{(0)}} + \frac{\mel{n}{\hat{V}}{n}}{(E - E_n^{(0)})^2}
\end{equation}
这样读出奇点就容易了:
\begin{equation}
E_n \approx E_n^{(0)} + \mel{n}{\hat{V}}{n}
\end{equation}
确实和扰动计算的结果一样!


我们再看二阶:
\begin{equation}
\hat{G}^R(E) \approx \hat{G}_0^R(E) + \hat{G}_0^R(E) \hat{V} \hat{G}_0^R(E) + \hat{G}_0^R(E) \hat{V} \hat{G}_0^R(E) \hat{V} \hat{G}_0^R(E)
\end{equation}
同样地, 我们计算矩阵元:
\begin{equation}
\mel{n}{\hat{G}^R(E)}{n} = \mel{n}{\hat{G}_0^R(E)}{n} + \mel{n}{\hat{G}_0^R(E) \hat{V} \hat{G}_0^R(E)}{n} + \mel{n}{\hat{G}_0^R(E) \hat{V} \hat{G}_0^R(E) \hat{V} \hat{G}_0^R(E)}{n}
\end{equation}
前两项我们已经算过了, 现在我们算第三项:
\begin{equation}
\mel{n}{\hat{G}_0^R(E) \hat{V} \hat{G}_0^R(E) \hat{V} \hat{G}_0^R(E)}{n} = \sum_{m} \frac{\mel{n}{\hat{V}}{m} \mel{m}{\hat{V}}{n}}{(E - E_n^{(0)})(E - E_m^{(0)})(E - E_n^{(0)})}
\end{equation}
整理一下:
\begin{equation}
\mel{n}{\hat{G}^R(E)}{n} = \frac{1}{E - E_n^{(0)}} + \frac{\mel{n}{\hat{V}}{n}}{(E - E_n^{(0)})^2} + \sum_{m} \frac{|\mel{n}{\hat{V}}{m}|^2}{(E - E_n^{(0)})^2 (E - E_m^{(0)})}
\end{equation}
我们现在可以写成:
\begin{equation}
\mel{n}{\hat{G}^R(E)}{n} = \frac{1}{E-E_n^{(0)}} + \frac{1}{(E - E_n^{(0)})^2} \left[ \mel{n}{\hat{V}}{n} + \sum_{m\neq n} \frac{|\mel{n}{\hat{V}}{m}|^2}{E - E_m^{(0)}} \right]
\end{equation}
我们定义:
\begin{equation}
\Sigma^{(2)}_n(E) = \mel{n}{\hat{V}}{n} + \sum_{m\neq n} \frac{|\mel{n}{\hat{V}}{m}|^2}{E - E_m^{(0)}}
\end{equation}
自然就有:
\begin{equation}
\Delta E_n^{(2)} = \Sigma^{(2)}_n(E_n^{(0)}) = \mel{n}{\hat{V}}{n} + \sum_{m\neq n} \frac{|\mel{n}{\hat{V}}{m}|^2}{E_n^{(0)} - E_m^{(0)}}
\end{equation}
这个和我们之前学过的二阶微扰理论结果完全一致!
所以说, 通过格林函数的 Dyson 方程和 Born 级数, 我们完全可以重新导出微扰理论的结果.


我们现在来解释一下我们使用非微扰方法求解 Dyson 方程的时候的困难:
\begin{equation}
\hat{G}^R(E) =  \frac{1}{E - \hat{H}_0 - \hat{V} + \mathrm{i}\,\eta}
\end{equation}
看起来, 我们只需要做一件事: 就是找到算符 $E - \hat{H}_0 - \hat{V}$的逆, 然后这不就是我们学量子力学扰动理论时候遇到的问题吗?
如果投影到 $\hat{H}_0$ 的本征态空间上, 那么我们就需要求解:
\begin{equation}
\mqty[ E - E_0^{(0)} - V_{00} & -V_{01} & -V_{02} & \cdots \\
-V_{10} & E - E_1^{(0)} - V_{11} & -V_{12} & \cdots \\
-V_{20} & -V_{21} & E - E_2^{(0)} - V_{22} & \cdots \\
\vdots & \vdots & \vdots & \ddots ]^{-1}
\end{equation}
这个矩阵的维度是无限的, 除非 $\hat{H}_0$ 的本征态空间是有限维的, 否则我们根本没法求出这个矩阵的逆.
如果能够求出这个矩阵的逆, 我们还费什么劲去用 Dyson 方程和 Born 级数呢?
只有在极少数情况下, 我们才能够解析地求出这个矩阵的逆, 比如我们就是考虑 exact diagonalization 的时候, 那就是说我们截断了这个矩阵, 只考虑有限维的子空间, 这样我们就能求出这个矩阵的逆了.
或者, 还有很少的一些可积模型 (integrable model), 我们也能解析地求出这个矩阵的逆.
所以说, Dyson 方程和 Born 级数的非微扰求解, 实际上并没有比直接求解 Hamiltonian 的本征态和本征值更简单多少.