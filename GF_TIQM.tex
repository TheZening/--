%=========================
%=========================
\section{格林函数: 不含时的量子力学}
\label{Sec: TI Quantum Mechanics}


我们从量子力学的一些基本观点出发, 可观测量对应于 Hermitian 算符, 无论是无穷维希尔伯特空间还是有限维希尔伯特空间, 本征值都是实数.
而且, 不同本征值对应的本征态是正交归一的, 从而我们可以选取一组完备的本征态作为希尔伯特空间的基底满足:
\begin{equation}
\hat{A} \ket{n} = a_n \ket{n}, \quad \braket{m}{n} = \delta_{mn}, \quad \sum_n \ketbra{n} = \mathbb{1}
\end{equation}
对于任意向量 $\ket{\psi}$, 我们都可以把他展开在这个完备基底上:
\begin{equation}
\ket{\psi} = \sum_n c_n \ket{n}, \quad c_n = \braket{n}{\psi}
\end{equation}
所以原来的算符 $\hat{A}$ 在这个基底下的矩阵元为:
\begin{equation}
\hat{A} = \sum_{n=1}^{\infty} a_n \ketbra{n} \Rightarrow A_{mn} = \mel{m}{\hat{A}}{n} = a_n \delta_{mn}
\end{equation}


如果我们选择能量本征态作为完备基底 (就是 Hamltonian 的本征态), 从而我们有:
\begin{equation}
\hat{H} \ket{n} = E_n \ket{n}, \quad \braket{m}{n} = \delta_{mn}, \quad \sum_n \ketbra{n} = \mathbb{1}
\end{equation}
考虑到:
\begin{equation}
\phi_n(x) \equiv \braket{x}{n} \Rightarrow \int \phi_m^*(x) \phi_n(x) \dd{x} = \delta_{mn}, \quad \sum_n \phi_n(x) \phi_n^*(x') = \delta(x - x') 
\end{equation}


有了这些基础, 我们先不要着急, 回到矩阵的本征值问题 (甚至是有限维):
\begin{equation}
\hat{A} \ket{n} = a_n \ket{n}
\end{equation}
然后我们考虑一个复数 $z$, 我们构造一个新的矩阵:
\begin{equation}
z\mathbb{1} - \hat{A}
\end{equation}
如果 $z\mathbb{1}-\hat{A}$ 是可逆的, 那么我们可以定义一个叫做 Resolvent 的矩阵:
\begin{equation}
\hat{R}(z) := (z\mathbb{1} - \hat{A})^{-1}
\end{equation}
\newterm{这个 $R$ 叫做 $A$ 在点 $z$ 的 Resolvent}.
所以对于所有可能的 $z$, 要么 $z\mathbb{1}-\hat{A}$ 是不可逆的 (就是谱 spectrum 上的点), 要么 $z\mathbb{1}-\hat{A}$ 是可逆的 (就是谱之外的点).

我们有:
\begin{equation}
(z\mathbb{1} - \hat{A}) \hat{R}(z) = \mathbb{1}
\end{equation}
在 $\hat{A}$ 的本征态基底下, 我们有:
\begin{equation}
z\mathbb{1} - \hat{A} = \mathrm{diag}(z - a_1, z - a_2, \cdots, z - a_n)
\end{equation}
这个逆还是好求一点:
\begin{equation}
\hat{R}(z) = \mathrm{diag}\left( \frac{1}{z - a_1}, \frac{1}{z - a_2}, \cdots, \frac{1}{z - a_n} \right)
\end{equation}
翻译成对角矩阵的形式:
\begin{equation}
\hat{R}(z) = \sum_{n=1}^{\infty} \frac{1}{z - a_n} \ketbra{n}
\end{equation}
所以我们看到, Resolvent 的奇点就是算符 $\hat{A}$ 的本征值.
这就是我们为什么说 Resolvent 可以揭示算符的谱的原因.


如果我们考虑的是 Hamiltonian $\hat{H}$, 那我们可以同时有束缚态和散射态, 也就是说谱是离散的和连续的混合, 所以我们有:
\begin{equation}
\hat{R}(z) = \sum_{n} \frac{1}{z - E_n} \ketbra{n} + \int \frac{1}{z - E} \ketbra{E} \dd{E}
\end{equation}
现在我们把 Resolvent 和格林函数联系起来, 首先我们注意到:
\begin{equation}
(z\mathbb{1} - \hat{H}) \hat{R}(z) = \mathbb{1}
\end{equation}
我们在坐标表象下写出来:
\begin{equation}
\mel{x}{(z\mathbb{1} - \hat{H}) \hat{R}(z)}{x'} = \mel{x}{\mathbb{1}}{x'} = \delta(x - x')
\end{equation}
我们把左边展开:
\begin{equation}
\int \mel{x}{(z\mathbb{1} - \hat{H})}{x''} \mel{x''}{\hat{R}(z)}{x'} \dd{x''} = \delta(x - x')
\end{equation}
计算这个积分, 我们有:
\begin{equation}
\int (z \delta(x - x'') - H(x, x'')) \mel{x''}{\hat{R}(z)}{x'} \dd{x''} = \delta(x - x')
\end{equation}
如果 Hamiltonian 是局域的, 也就是说:
\begin{equation}
H(x, x'') = \delta(x - x'') \hat{H}_x
\end{equation}
那么我们有:
\begin{equation}
(z - \hat{H}_x) \mel{x}{\hat{R}(z)}{x'} = \delta(x - x')
\end{equation}
仔细观察这个方程, 我们发现他和格林函数的定义方程是一样的!
\begin{equation}
G(x, x'; z) = \mel{x}{\hat{R}(z)}{x'} = \mel{x}{(z\mathbb{1} - \hat{H})^{-1}}{x'}
\end{equation}
所以在这个角度下, Schrödinger 方程的格林函数就是 Hamiltonian 的 Resolvent 在坐标表象下的矩阵元.


现在我们不得不面临一个灾难级的翻译了, 在线性代数中, 我们一说到 Kernel (核), 大家第一反应就是想到线性映射的核空间 (Null Space), 也就是所有被映射到零向量的向量构成的子空间.
但实际上, 我们在讨论积分方程和变换的时候, Kernel 指的就是积分核, \newterm{算符在某个表象下的核 = 这个算符在这组基底上的矩阵元}.
比如我们考虑:
\begin{equation}
\mel{x}{\hat{A}}{\psi} = \int \mel{x}{\hat{A}}{x'} \braket{x'}{\psi} \dd{x'} = \int A(x, x') \psi(x') \dd{x'}
\end{equation}
这里的 $A(x, x') = \mel{x}{\hat{A}}{x'}$ 就是算符 $\hat{A}$ 在坐标表象下的核 (Kernel).
所以我们必须小心区分这两个概念.


所以从这角度上讲, 格林函数就是 Hamiltonian 的 Resolvent 在坐标表象下的核:
\begin{equation}
G(x, x'; z) = \mel{x}{(z\mathbb{1} - \hat{H})^{-1}}{x'}
\end{equation}
我们也可以把他写成本征态展开的形式:
\begin{equation}
G(x, x'; z) = \sum_{n} \frac{\phi_n(x) \phi_n^*(x')}{z - E_n} + \int \frac{\phi_E(x) \phi_E^*(x')}{z - E} \dd{E}
\end{equation}


现在我们考虑一个算符方程(先不说什么能量, Hamiltonian 之类的):
\begin{equation}
(z\mathbb{1} - \hat{H}) \ket{\psi} = \ket{\phi}
\end{equation}
在位置表象下, 我们有:
\begin{equation}
(z-\hat{H}_x) \psi(x) = \phi(x)
\end{equation}
考虑到:
\begin{equation}
R(z) = (z\mathbb{1} - \hat{H})^{-1}
\end{equation}
我们自然可以写出来:
\begin{equation}
\ket{\psi} = R(z) \ket{\phi}
\end{equation}
取位置表象:
\begin{equation}
\psi(x) = \mel{x}{R(z)}{\phi} = \int \mel{x}{R(z)}{x'} \braket{x'}{\phi} \dd{x'} = \int G(x, x'; z) \phi(x') \dd{x'}
\end{equation}
这就是我们一直在说的格林函数方法: 我们把原来的算符方程转化成一个积分方程, 积分核就是格林函数.
\begin{equation}
\psi(x) = \int G(x, x'; z) \phi(x') \dd{x'}
\end{equation}


现在我们把这些数学问题翻译回物理问题.
我们考虑本征值问题:
\begin{equation}
\hat{H} \ket{n} = E_n \ket{n}
\end{equation}
从线性代数的角度来看, 本征值的一个等价定义是:
\begin{equation}
(E_n \mathbb{1} - \hat{H}) \ket{n} = 0
\end{equation}
也就是说, 当 $z = E_n$ 的时候, $z\mathbb{1} - \hat{H}$ 是不可逆的, 也就是说 Resolvent 在 $z = E_n$ 有奇点.
就是说 $z\mathbb{1} - \hat{H}$ 在 $z = E_n$ 的零空间 (Null Space) 非平凡.
所以能量本征值就是 Hamiltonian 的谱, 也是 Hamiltonian 的 Resolvent 的奇点所在, 也就是格林函数的奇点所在.
这就是我们为什么说格林函数揭示了系统的能谱结构的原因.

现在我们至少明确了一个问题: 我们实际上是求的算符 $(z\mathbb{1} - \hat{H})$ 的格林函数, 有的时候你看到文献上所谓的 ``Hamiltonian 的格林函数'', 其实就是指的这个算符的格林函数, 是省略的称呼.


可是我们为什么要带着这个 $z$ 呢?
这个问题的答案在于, 不是所有的 Hamiltonian 都有良好的逆算符, 比如有0本征值的 Hamiltonian, 为了避免这个逆算符不存在的问题, 我们引入了一个复数 $z$, 取值任意, 从而达到一般意义下 $(z\mathbb{1} - \hat{H})$ 可逆的目的.
这样的话谱点就被反应在了格林函数的奇点上, 而不是 Hamiltonian 的逆算符不存在上.


把 $z$ 取成复数还有另外一个好处, 就是我们可以通过调节 $z$ 的虚部来控制格林函数的收敛性, 以及容纳因果关系 (等价于施加自然的边界条件).
我们现在一次说清楚, 然后逐步分析: 
\begin{equation}
\text{频域取}+\mathrm{i}\,\epsilon = \text{时域带}\theta(t-t') = \text{推迟格林函数} = \text{自然因果关系}
\end{equation}
我们先来从数学上解释一下这个问题, 我们考虑如下的恒等式:
\begin{equation}
\frac{1}{x+\mathrm{i}\,\epsilon} = -\mathrm{i}\, \int_{0}^{+\infty} \mathrm{e}^{\mathrm{i}\,(x+\mathrm{i}\,\epsilon) t} \dd{t}, \quad \epsilon > 0
\end{equation}
我们把积分拆开看:
\begin{equation}
\int_{0}^{+\infty} \mathrm{e}^{\mathrm{i}\,x t} \mathrm{e}^{-\epsilon t} \dd{t}
\end{equation}
只有当 $\epsilon > 0$ 的时候, 指数项 $\mathrm{e}^{-\epsilon t}$ 才能保证积分收敛.
现在我们对齐了第一个拼图: 当我们在频域格林函数中取 $z + \mathrm{i}\,\epsilon$ 的时候, 我们保证了格林函数的积分收敛.
我们做一个推广:
\begin{equation}
(z-\hat{H} + \mathrm{i}\,\epsilon)^{-1} = -\mathrm{i} \int_{0}^{+\infty} \mathrm{e}^{\mathrm{i}\,(z-\hat{H} + \mathrm{i}\,\epsilon) t} \dd{t}, \quad \epsilon > 0
\end{equation}

现在我们来看第二个问题, 为什么 $+\mathrm{i}\,\epsilon$ 对应于 retarded 推迟格林函数 (就是 $\theta(t-t')$ 出现的格林函数).
我们考虑时域的格林函数的 Fourier 逆变换:
\begin{equation}
G^R(t) = \frac{1}{2\pi} \int_{-\infty}^{+\infty} \tilde{G}^R(\omega) \mathrm{e}^{-\mathrm{i}\,\omega t} \dd{\omega}
\end{equation}
这个时候又有一个记号上必须明确的地方了.
\begin{enumerate}
\item 我们之前推导的是: 能量域格林函数 $\tilde{G}(z)$ 是Resolvent在坐标表象下的矩阵元 (我们现在给这个格林函数加了上标).
\item 如果我们不取坐标的矩阵元, 我们就成为了算符形式的格林函数 $\tilde{G}(z) = (z\mathbb{1} - \hat{H})^{-1}$.
以后我们管他叫\newterm{能量域格林函数}.
\item 然后我们说 $z$ 是个复数对吧, 现在把他规定成一个实数 $E$ 加上一个虚部 $+\mathrm{i}\,\epsilon$, 这个 $E$ 就是我们常说的能量变量, 不是真的能量, 就是个符号 (此时说明 $z$ 在复平面上半平面).
从而我们的格林函数变成了:
\begin{equation}
\tilde{G}^R(E) = (E + \mathrm{i}\,\epsilon - \hat{H})^{-1}
\end{equation}
我们把这个叫做\newterm{能量延迟格林函数} energy domain retarded Green function.
\item 如果我们把 $z$ 取成 $E - \mathrm{i}\,\epsilon$ (此时说明 $z$ 在复平面下半平面).
那么我们就叫做\newterm{能量预先格林函数} energy domain advanced Green function:
\begin{equation}
\tilde{G}^A(E) = (E - \mathrm{i}\,\epsilon - \hat{H})^{-1}
\end{equation}
\item 如果我们不使用能量作为变量, 而是使用频率 $\omega$, 那么我们就把能量域格林函数写成:
\begin{equation}
\tilde{G}^{(R/A)}(\omega) = (\hbar \omega \pm \mathrm{i}\,\epsilon - \hat{H})^{-1}
\end{equation}
他的量纲还是$[\text{能量}]^{-1}$.
\item 还有很多文献喜欢直接让 $\epsilon \to 0^+$, 也就是直接写成:
\begin{equation}
\tilde{G}^{(R/A)}(E) = (E \pm \mathrm{i}\,0^+ - \hat{H})^{-1}
\end{equation}
这些我们会交叉使用.
\end{enumerate}


现在我们就以上面的记号为基础, 来解释为什么 $+\mathrm{i}\,\epsilon$ 对应于推迟格林函数.
我们考虑能量域延迟格林函数:
\begin{equation}
\tilde{G}^R(E) = \frac{1}{E - \hat{H} + \mathrm{i}\,\epsilon}
\end{equation}
这个时候还是算符, 我们取一个能级出来, 比如就是能量本征态 $\ket{n}$:
\begin{equation}
\tilde{G}^R_{n}(E) = \mel{n}{\tilde{G}^R(E)}{n} = \frac{1}{E - E_n + \mathrm{i}\,\epsilon}
\end{equation}
我们先看这一个能量本征态怎么办, 然后等下在用谱分解方法把所有能级都考虑进来.
我们对这个能量本征态的格林函数做 Fourier 逆变换:
\begin{equation}
G^R_{n}(t) = \frac{1}{2\pi} \int_{-\infty}^{+\infty} \frac{1}{E - E_n + \mathrm{i}\,\epsilon} \mathrm{e}^{-\mathrm{i}\,E t/\hbar} \dd{E}
\end{equation}
我们现在考虑 $t>0$ 的情况, 重点关注:
\begin{equation}
\mathrm{e}^{-\mathrm{i}\,E t/\hbar} \quad t>0
\end{equation}
我们首先考虑解析延拓, 然后为了让围道远处的积分消失 (就是 $E \to +\infty$ 的时候), 我们必须要让 $\Im E < 0$.
也就是说, 我们必须把积分路径闭合在下半平面.
这时候再看极点的位置, 极点在:
\begin{equation}
E = E_n - \mathrm{i}\,\epsilon
\end{equation}
这个极点在下半平面, 所以我们把积分路径闭合在下半平面的时候 (注意啊, 下半平面, 留数带个负号了), 极点被包围住了.
根据留数定理, 我们有:
\begin{equation}
G^R_{n}(t) = \frac{1}{2\pi} ( -2\pi \mathrm{i} ) \mathrm{e}^{-\mathrm{i}\,(E_n - \mathrm{i}\,\epsilon) t/\hbar} = -\mathrm{i} \mathrm{e}^{-\mathrm{i}\,E_n t/\hbar} \mathrm{e}^{-\epsilon t/\hbar}, \quad t>0
\end{equation}
此时再取极限:
\begin{equation}
\lim_{\epsilon \to 0^+} G^R_{n}(t) = -\mathrm{i} \mathrm{e}^{-\mathrm{i}\,E_n t/\hbar}, \quad t>0
\end{equation}
没问题了, 继续看 $t<0$ 的情况, 这时候我们关注:
\begin{equation}
G^R_{n}(t) = \frac{1}{2\pi} \int_{-\infty}^{+\infty} \frac{1}{E - E_n + \mathrm{i}\,\epsilon} \mathrm{e}^{-\mathrm{i}\,E t/\hbar} \dd{E}
\end{equation}
我们考虑 $t<0$ 的时候, 指数项变成:
\begin{equation}
\mathrm{e}^{-\mathrm{i}\,E t/\hbar} \quad t<0
\end{equation}
为了让围道远处的积分消失, 我们必须要让 $\Im E > 0$.
也就是说, 我们必须把积分路径闭合在上半平面.
这时候再看极点的位置, 极点在:
\begin{equation}
E = E_n - \mathrm{i}\,\epsilon
\end{equation}
这个极点在下半平面, 所以我们把积分路径闭合在上半平面的时候, 极点没有被包围住了.
根据留数定理, 我们直接用Cauchy积分定理, 我们有:
\begin{equation}
G^R_{n}(t) = 0, \quad t<0
\end{equation}
综上所述, 我们有:
\begin{equation}
G^R_{n}(t) = -\mathrm{i} \Theta(t) \mathrm{e}^{-\mathrm{i}\,E_n t/\hbar}
\end{equation}
这就是推迟格林函数的形式, 我们看到, 频域格林函数中取 $+\mathrm{i}\,\epsilon$ 确实对应于时域格林函数中的 $\Theta(t)$ 因果关系.


\reminder{重要提示}: 以上的推导我们必须假设 Fourier 变换的定义是:
\begin{equation}
\tilde{G}(\omega) = \int_{-\infty}^{+\infty} G(t) \mathrm{e}^{\mathrm{i}\,\omega t} \dd{t}, \quad G(t) = \frac{1}{2\pi} \int_{-\infty}^{+\infty} \tilde{G}(\omega) \mathrm{e}^{-\mathrm{i}\,\omega t} \dd{\omega}
\end{equation}
如果你定义成相反的符号, 那么 $+\mathrm{i}\,\epsilon$ 就对应于预先格林函数了.


现在我们把所有能级都考虑进来, 我们有:
\begin{equation}
\tilde{G}^R(E) = \sum_{n} \frac{1}{E - E_n + \mathrm{i}\,\epsilon} \ketbra{n}
\end{equation}
整体做 Fourier 逆变换:
\begin{equation}
G^R(t) = \frac{1}{2\pi} \int_{-\infty}^{+\infty} \tilde{G}^R(E) \mathrm{e}^{-\mathrm{i}\,E t/\hbar} \dd{E}
\end{equation}
我们对每一个能级都做同样的分析, 最终我们得到:
\begin{equation}
G^R(t) = \frac{1}{2\pi}  \sum_n \ketbra{n} \int_{-\infty}^{+\infty} \frac{1}{E - E_n + \mathrm{i}\,\epsilon} \mathrm{e}^{-\mathrm{i}\,E t/\hbar} \dd{E} = -\mathrm{i} \Theta(t) \sum_n \ketbra{n} \mathrm{e}^{-\mathrm{i}\,E_n t/\hbar}
\end{equation}
最终我们有:
\begin{equation}
G^R(t) = -\mathrm{i} \Theta(t) \mathrm{e}^{-\mathrm{i}\,\hat{H} t/\hbar}
\end{equation}
以及:
\begin{equation}
G^R(t-t') = -\mathrm{i} \Theta(t-t') \mathrm{e}^{-\mathrm{i}\,\hat{H} (t-t')/\hbar}
\end{equation}
现在我们重点强调一下单位问题:
\begin{enumerate}
\item 能量域格林函数 $\tilde{G}^R(E) = (E + \mathrm{i}\,\epsilon - \hat{H})^{-1}$ 的量纲是 $[\text{能量}]^{-1}$.
\item 频域格林函数 $\tilde{G}^R(\omega) = (\hbar \omega + \mathrm{i}\,\epsilon - \hat{H})^{-1}$ 的量纲也是 $[\text{能量}]^{-1}$, 只不过我们用$\omega$代替了$E/\hbar$.
\item 时域格林函数 $G^R(t-t') = -\mathrm{i} \theta(t-t') \mathrm{e}^{-\mathrm{i}\,\hat{H} (t-t')/\hbar}$ 是无量纲的.
\item Heaviside阶跃函数, $\Theta(t-t')$ 是无量纲的.
\item $\delta$函数 $\delta(t-t')$ 的量纲是 $[\text{时间}]^{-1}$:
\begin{equation}
\int_{-\infty}^{+\infty} \delta(t-t') \dd{t'} = 1 \quad \dv{\Theta(t-t')}{t} = \delta(t-t')
\end{equation} 
\end{enumerate}


我们现在考虑一下一维自由粒子的格林函数.
他的 Hamiltonian 为:
\begin{equation}
\hat{H} = \frac{\hat{p}^2}{2m} = -\frac{\hbar^2}{2m} \pdv[2]{}{x}
\end{equation}
他的格林函数为:
\begin{equation}
(z-H_x) \tilde{G}(x, x'; z) = \delta(x-x')
\end{equation}
我们代入 Hamiltonian:
\begin{equation}
\left( z + \frac{\hbar^2}{2m} \pdv[2]{}{x} \right) \tilde{G}(x, x'; z) = \delta(x-x')
\end{equation}
引入如下记号:
\begin{equation}
k^2 := \frac{2m z}{\hbar^2} \Rightarrow z = \frac{\hbar^2 k^2}{2m}
\end{equation}
我们有:
\begin{equation}
\left( \pdv[2]{}{x} + k^2 \right) G(x, x'; z) = \frac{2m}{\hbar^2} \delta(x-x')
\end{equation}
注意啊, 这个时候我们考虑的不是时间问题了, 是一个空间问题.
这个翻译成自然的条件就是一开始波在 $x'$ 处被激发出来, 然后向两边传播出去.
对于 $x > x'$ 的区域, 波自然是向右传播的, 对于 $x < x'$ 的区域, 波自然是向左传播的.
所以我们猜测格林函数的形式为:
\begin{equation}
\tilde{G}(x, x'; z) = \begin{cases} C_- \mathrm{e}^{-\mathrm{i}\,k (x - x')}, & x < x' \\ C_+ \mathrm{e}^{\mathrm{i}\,k (x - x')}, & x > x' \end{cases}
\end{equation}
连续性要求我们有:
\begin{equation}
C_- = C_+ := C
\end{equation}
然后我们对方程两边在 $x = x'$ 处做积分:
\begin{equation}
\dv{G}{x} \Big|_{x=x'+0} - \dv{G}{x} \Big|_{x=x'-0} = \frac{2m}{\hbar^2} = 2\mathrm{i}\, k C
\end{equation}
从而我们得到:
\begin{equation}
C = \frac{m}{\mathrm{i}\, \hbar^2 k}
\end{equation}
所以我们最终得到一维自由粒子的格林函数为:
\begin{equation}
\tilde{G}(x, x'; z) = \frac{m}{\mathrm{i}\, \hbar^2 k} \begin{cases} \mathrm{e}^{-\mathrm{i}\,k (x - x')}, & x < x' \\ \mathrm{e}^{\mathrm{i}\,k (x - x')}, & x > x' \end{cases} 
\end{equation}
我们也可以写成:
\begin{equation}
\tilde{G}(x, x'; z) = \frac{m}{\mathrm{i}\, \hbar^2 k} \mathrm{e}^{\mathrm{i}\,k |x - x'|} \quad k=\sqrt{\frac{2m z}{\hbar^2}}
\end{equation}
这个格林函数描述了一个在 $x'$ 处被激发出来的波, 然后向两边传播出去.
这个就是一维自由粒子的格林函数, 我们已经投影到了位置表象下:
\begin{equation}
\tilde{G}(x, x'; z) = \mel{x}{(z\mathbb{1} - \hat{H})^{-1}}{x'} = \frac{m}{\mathrm{i}\, \hbar^2 k} \mathrm{e}^{\mathrm{i}\,k |x - x'|} \quad k=\sqrt{\frac{2m z}{\hbar^2}}
\end{equation}
如果我们取 $z = E + \mathrm{i}\,\epsilon$, 那么我们就得到了能量延迟格林函数:
\begin{equation}
\tilde{G}^R(x, x'; E) = \mel{x}{(E + \mathrm{i}\,\epsilon - \hat{H})^{-1}}{x'} = \frac{m}{\mathrm{i}\, \hbar^2 k} \mathrm{e}^{\mathrm{i}\,k |x - x'|} \quad k=\sqrt{\frac{2m (E + \mathrm{i}\,\epsilon)}{\hbar^2}}
\end{equation}

可能刚开始学习的时候会有一个困惑: 这个能还原到我们之前的那种 $1/...$ 的形式吗?
我们来看一下, 先写出来:
\begin{equation}
\tilde{G}^R(E) = \frac{1}{E + \mathrm{i}\,\epsilon - \hat{H}}
\end{equation}
留神了! 这是一个连续谱问题:
\begin{equation}
\tilde{G}^R(E) = \int \frac{\ketbra{E_1}}{E + \mathrm{i}\,\epsilon - E'} \dd{E_1}
\end{equation}
我们取位置表象:
\begin{equation}
\tilde{G}^R(x, x'; E) = \mel{x}{\tilde{G}^R(E)}{x'} = \int \frac{1}{E + \mathrm{i}\,\epsilon - E_1} \phi_{E_1}(x) \phi_{E_1}^*(x') \dd{E_1}
\end{equation}
这里的 $\phi_{E_1}(x) = \braket{x}{E_1}$ 是能量本征态:
\begin{equation}
\phi_{E_1}(x) = \frac{1}{\sqrt{2\pi}} \mathrm{e}^{\mathrm{i}\,k_1 x}, \quad E_1 = \frac{\hbar^2 k_1^2}{2m}
\end{equation}
所以我们有:
\begin{equation}
\tilde{G}^R(x, x'; E) = \frac{1}{2\pi} \int_{-\infty}^{+\infty} \frac{\mathrm{e}^{\mathrm{i}\,k_1 (x - x')}}{E + \mathrm{i}\,\epsilon - \hbar^2 k_1^2 / 2m} \dd{k_1}
\end{equation}
做这个积分还是应用留数定理:
\begin{equation}
\tilde{G}^R(x, x'; E) = \frac{m}{\mathrm{i}\, \hbar^2 k} \mathrm{e}^{\mathrm{i}\,k |x - x'|} \quad k=\sqrt{\frac{2m (E + \mathrm{i}\,\epsilon)}{\hbar^2}}
\end{equation}
这样就走通了:
\begin{equation}
\tilde{G}^R(x, x'; E) = \mel{x}{(E + \mathrm{i}\,\epsilon - \hat{H})^{-1}}{x'}
\end{equation}


现在我们看看时域格林函数:
\begin{equation}
G^R(x,t; x', t') = \frac{1}{2\pi} \int_{-\infty}^{+\infty} \tilde{G}^R(x, x'; E) \mathrm{e}^{-\mathrm{i}\,E (t-t')/\hbar} \dd{E}
\end{equation}
代入能量延迟格林函数:
\begin{equation}
G^R(x,t; x', t') = \frac{1}{2\pi} \int_{-\infty}^{+\infty} \frac{m}{\mathrm{i}\, \hbar^2 k} \mathrm{e}^{\mathrm{i}\,k |x - x'|} \mathrm{e}^{-\mathrm{i}\,E (t-t')/\hbar} \dd{E}
\end{equation}
做这个积分, 我们使用 $k$ 做积分也行, 也可以直接用 $E$ 做积分, 用 $k$ 做积分的话稍微简单一些:
\begin{equation}
E = \frac{\hbar^2 k^2}{2m} \Rightarrow \dd{E} = \frac{\hbar^2 k}{m} \dd{k}
\end{equation}
这一类 Gauss 积分我们已经很熟悉了, 最终我们得到:
\begin{equation}
G^R(x,t; x', t') = \frac{1}{\mathrm{i}} \Theta(t-t') \sqrt{\frac{m}{2\pi \mathrm{i}\, \hbar (t-t')}} \exp \left[ \frac{\mathrm{i}\, m |x - x'|^2}{2\hbar (t-t')} \right]
\end{equation}
单位是 $[\text{长度}]^{-1}$.
检查一下单位: $\tilde{G}^R(E)$的单位是 $[\text{能量}]^{-1}$, 投影到矩阵元上是 $[\text{能量}]^{-1}\times[\text{长度}]^{-1}$, 做 Fourier 逆变换的时候乘以 $\dd{E}$, 所以时域格林函数的单位是 $[\text{长度}]^{-1}$, 和上面的结果是一致的.


现在我们可以完成我们拼图中的最后一环了: 格林函数和时间演化算符之间的关系.
我们现在求的粒子, 实际上是一个自由粒子, 我们从来没有提到什么时间演化的问题.
对于不含时的问题, 我们只能求到能量本征态, 但是我们并不知道系统是如何随时间演化的, 同时我们引入了 $z$ 这个复数, 使得我们求的是 $(z\mathbb{1} - \hat{H})$ 的逆算符.
现在我们把这些问题联系起来, 我们考虑我们不含时的 Hamiltonian 对应的含时 Schrödinger 方程:
\begin{equation}
\mathrm{i}\, \hbar \pdv{t} \ket{\psi(t)} = \hat{H} \ket{\psi(t)}
\end{equation}
我们注意到他对应的格林函数要满足:
\begin{equation}
\left( \mathrm{i}\, \hbar \pdv{t} - \hat{H} \right) G^R = \delta(t-t') \mathbb{1}
\end{equation}
此时我们为什么又不需要 $z$ 了呢?
因为我们现在考虑的是时间演化的问题, 我们可以通过边界条件来唯一确定解, 所以不需要再引入 $z$ 这个复数来保证逆算符存在了.
我们的边界条件就是自然因果关系, 也就是 $\Theta(t-t')$ (对应与频域的 $+\mathrm{i}\,\epsilon$).

我们可以求解这个方程, 重复之前的标准格林函数, 我们可以发现:
\begin{equation}
G^R(t-t') = \frac{1}{\mathrm{i}\,\hbar} \Theta(t-t') \mathrm{e}^{-\mathrm{i}\, \hat{H} (t-t')/\hbar}
\end{equation}
对于不含时 Hamiltonian, 时间演化算符就是:
\begin{equation}
\hat{U}(t, t') = \mathrm{e}^{-\mathrm{i}\, \hat{H} (t-t')/\hbar}
\end{equation}
所以我们最终得到:
\begin{equation}
G^R(t-t') = \frac{1}{\mathrm{i}\,\hbar} \Theta(t-t') \hat{U}(t, t')
\end{equation}
如果把他投影到位置空间, 我们有:
\begin{equation}
G^R(x,t; x', t') = \frac{1}{\mathrm{i}\,\hbar} \Theta(t-t') \mel{x}{\hat{U}(t, t')}{x'}
\end{equation}
\newterm{传播子} (Propagator) 就是时间演化算符在位置表象下的矩阵元:
\begin{equation}
K(x, t; x', t') := \mel{x}{\hat{U}(t, t')}{x'}
\end{equation}
所以我们有:
\begin{equation}
G^R(x,t; x', t') = \frac{1}{\mathrm{i}\,\hbar} \Theta(t-t') K(x, t; x', t')
\end{equation}
这就是格林函数和时间演化算符之间的关系.
我们完全可以对于自由粒子模型来验证一下这个关系, 结果是:
\begin{equation}
K(x, t; x', t') = \sqrt{\frac{m}{2\pi \mathrm{i}\, \hbar (t-t')}} \exp \left[ \frac{\mathrm{i}\, m |x - x'|^2}{2\hbar (t-t')} \right]
\end{equation}
这个时候我们要注意了, 我们之前求的时域格林函数和这个差了一个 $\hbar$ 的因子, 这个原则上取哪一个都可以, 只是习惯问题, 但是因为这个从时间演化算符推导出来的关系更为基础, 所以我们建议大家以后都采用这个定义:
\begin{equation}
G^R(t-t') = \frac{1}{\mathrm{i}\,\hbar} \Theta(t-t') \mathrm{e}^{-\mathrm{i}\, \hat{H} (t-t')/\hbar}
\end{equation}
以及从频域变回时域的时候, 也要注意加上 $\hbar$:
\begin{equation}
G(t-t') = \frac{1}{2\pi \hbar} \int_{-\infty}^{+\infty} \tilde{G}(E) \mathrm{e}^{-\mathrm{i}\,E (t-t')/\hbar} \dd{E}
\end{equation}

 
