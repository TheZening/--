%=========================
%=========================
\section{格林函数: 谱密度}
\label{Sec: Spectral density}


我们还是从量子力学出发, 考虑一个体系的 Hamiltonian $\hat{H}$ 及其本征值问题
\begin{equation}
\hat{H} \ket{n} = E_n \ket{n}
\end{equation}
我们就先假设这个谱是离散的, 后面再讨论连续谱的情况.
如果随便拿一个态, 比如 $\ket{\psi}$, 那么它可以展开为本征态的线性组合
\begin{equation}
\ket{\psi} = \sum_n c_n \ket{n}
\end{equation}
其中 $c_n = \braket{n}{\psi}$.
那么这个态的能量期望值就是
\begin{equation}
\mel{\psi}{\hat{H}}{\psi} = \sum_n |c_n|^2 E_n
\end{equation}
也就是说, 我们测的能量为 $E_n$ 的概率就是 $|c_n|^2$.
\reminder{一个重要的提示, 我们这里的 $n$ 是量子态的编号, 不是能量的编号!}
这就引出了\newterm{谱密度 (spectral density)} 的概念.


\begin{mdefinition}{谱密度, spectral density}
设 $\hat{H}$ 的本征值为 $E_n$, 本征态为 $\ket{n}$, 则态 $\ket{\psi}$ 的谱密度算符为:
\begin{equation}
\hat{\rho}(E) = \delta(E - \hat{H}) = \sum_n \delta(E - E_n) \ketbra{n}
\end{equation}
态 $\ket{\psi}$ 在能量 $E$ 处的谱密度定义为: 对于态 $\ket{\psi}$ 进行一次能量测量, 测得能量为 $E$ 的概率密度为:
\begin{equation}
S_{\ket{\psi}}(E) = \mel{\psi}{\hat{\rho}(E)}{\psi} = \sum_n |\braket{n}{\psi}|^2 \delta(E - E_n)
\end{equation}
\end{mdefinition}


便于我们理解的说法, 就是谱密度告诉我们, 对于任意一个态, 它在不同能量处的“分量”有多大.
而这个 $\delta$ 函数, 就明确的筛选出来了, 必须是本征值对应的能量才有贡献.
谱密度的归一化条件是
\begin{equation}
\int_{-\infty}^{+\infty} S_{\ket{\psi}}(E) \dd{E} = 1
\end{equation}
因为我们总是能测到某个能量值.


我们举个例子, 你手里拿着一个三棱镜, 让阳光通过它, 你会看到七彩的光谱.
这是因为阳光是由很多不同频率的光混合而成的, 三棱镜把不同频率的光分离开来, 你就能看到不同频率的光了.
然后你对着光谱, 拿着一个光度计, 测量每个频率处的光强.
你测得的光强分布, 就类似于谱密度: 它告诉你, 在不同频率处, 光的“分量”有多大.
如果你测得的光强在某个频率处特别大, 那么说明, 这个频率的光在阳光中占有很大的比例.

同样的,


现在我们转头补充一点数学上的内容.
\begin{mtheorem}{Sokhotski–Plemelj 定理}
对于任意实数 $x$, 有
\begin{equation}
\lim_{\eta \to 0^+} \frac{1}{x \pm \mathrm{i}\,\eta} = \mathcal{P} \frac{1}{x} \mp \mathrm{i}\,\pi \delta(x)
\end{equation}
其中 $\mathcal{P}$ 表示主值积分 (Cauchy principal value), 即
\begin{equation}
\mathcal{P} \int_{-\infty}^{+\infty} \frac{f(x)}{x} \dd{x} = \lim_{\epsilon \to 0^+} \left( \int_{-\infty}^{-\epsilon} \frac{f(x)}{x} \dd{x} + \int_{\epsilon}^{+\infty} \frac{f(x)}{x} \dd{x} \right)
\end{equation}  
\end{mtheorem}
这个定理的使用一定是要在积分意义下的, 也就是说, 对于任意试函数 $f(x)$, 有
\begin{equation}
\lim_{\eta \to 0^+} \int_{-\infty}^{+\infty} \frac{f(x)}{x \pm \mathrm{i}\,\eta} \dd{x} = \mathcal{P} \int_{-\infty}^{+\infty} \frac{f(x)}{x} \dd{x} \mp \mathrm{i}\,\pi f(0)
\end{equation}
这个定理在处理含有 $\delta$ 函数的表达式时非常有用, 我们后面会经常用到它.


现在我们来说明 (\reminder{而不是严格证明}), 这个定理的来源.
我们一切的起源就是对于如下积分的计算尝试:
\begin{equation}
I = \lim_{\eta \to 0^+} \int_{-\infty}^{+\infty} \frac{f(x)}{x \pm \mathrm{i}\,\eta} \dd{x}
\end{equation}
这里 $f(x)$ 是一个在光滑的, 在无穷远处足够快趋于零的试验函数.
我们专注于正号的情况, 负号的情况类似.
我们可以把这个积分拆成两部分:
\begin{equation}
I = \lim_{\eta \to 0^+} \left[ \int_{-\infty}^{+\infty} \frac{x f(x)}{x^2 + \eta^2} \dd{x} - \mathrm{i}\, \int_{-\infty}^{+\infty} \frac{\eta f(x)}{x^2 + \eta^2} \dd{x} \right]
\end{equation}
我们先看第二部分:
\begin{equation}
I_2 = \lim_{\eta \to 0^+} \int_{-\infty}^{+\infty} \frac{\eta f(x)}{x^2 + \eta^2} \dd{x}
\end{equation}
这指示我们想到一类 Lorentzian 函数:
\begin{equation}
L_\eta(x) = \frac{1}{\pi} \frac{\eta}{x^2 + \eta^2}
\end{equation}
他满足:
\begin{equation}
\int_{-\infty}^{+\infty} L_\eta(x) \dd{x} = 1 \quad \text{且} \quad \lim_{\eta \to 0^+} L_\eta(x) = \delta(x)
\end{equation}
因此, 我们有:
\begin{equation}
\lim_{\eta \to 0^+} \frac{\eta}{x^2 + \eta^2} = \pi \delta(x)
\end{equation}
所以虚部的积分就是:
\begin{equation}
I_2 = \lim_{\eta \to 0^+} \int_{-\infty}^{+\infty} \frac{\eta f(x)}{x^2 + \eta^2} \dd{x} = \pi \int_{-\infty}^{+\infty} f(x) \delta(x) \dd{x} = \pi f(0)
\end{equation}
现在我们得看看这个实部的积分:
\begin{equation}
I_1 = \lim_{\eta \to 0^+} \int_{-\infty}^{+\infty} \frac{x f(x)}{x^2 + \eta^2} \dd{x}
\end{equation}
我们还是把积分拆成两部分:
\begin{equation}
I_1 =  \int_{|x| > \epsilon} \frac{f(x)}{x} \dd{x} + \lim_{\eta \to 0^+} \int_{-\epsilon}^{+\epsilon} \frac{x f(x)}{x^2 + \eta^2} \dd{x}
\end{equation}
其中 $\epsilon$ 是一个很小的正数, 而且 $\epsilon \gg \eta$.
对于第一部分, 是我们直接取极限 $\eta \to 0^+$ 得到的, 这个时候 $x$ 大于 $\epsilon$, 所以分母里的 $\eta^2$ 可以忽略不计, 而且也不会遇到 $x=0$ 的奇点.
对于第二部分, 我们注意到, 当 $\eta \to 0^+$ 时, $f(x)$ 在 $[-\epsilon, +\epsilon]$ 上可以近似看作常数 $f(0)$, 因为 $f(x)$ 是光滑的, 而且:
\begin{equation}
\frac{x}{x^2 + \eta^2}
\end{equation}
是一个奇函数, 所以第二部分的积分为零.
因此, 我们得到了:
\begin{equation}
I_1 = \int_{-\infty}^{+\infty} \frac{x f(x)}{x^2+\eta^2} \dd{x} = \mathcal{P} \int_{-\infty}^{+\infty} \frac{f(x)}{x} \dd{x}
\end{equation}
所以最后我们还是有:
\begin{equation}
I = \lim_{\eta \to 0^+} \int_{-\infty}^{+\infty} \frac{f(x)}{x + \mathrm{i}\,\eta} \dd{x} = \mathcal{P} \int_{-\infty}^{+\infty} \frac{f(x)}{x} \dd{x} - \mathrm{i}\,\pi f(0)
\end{equation}
这就证明了 Sokhotski–Plemelj 定理:
\begin{equation}
\lim_{\eta \to 0^+} \frac{1}{x + \mathrm{i}\,\eta} = \mathcal{P} \frac{1}{x} - \mathrm{i}\,\pi \delta(x)
\end{equation}
所以我们还可以写出来:
\begin{equation}
\lim_{\eta \to 0^+} \frac{1}{x - \mathrm{i}\,\eta} = \mathcal{P} \frac{1}{x} + \mathrm{i}\,\pi \delta(x)
\end{equation}
还可以略略变形:
\begin{equation}
\lim_{\eta \to 0^+} \frac{1}{(x-y) \pm \mathrm{i}\,\eta} = \mathcal{P} \frac{1}{x-y} \mp \mathrm{i}\,\pi \delta(x-y)
\end{equation}


\begin{mexample}{$\Theta$ 函数的积分表示}
我们之前提到过:
\begin{equation}
\Theta(t) = -\frac{1}{2\pi \mathrm{i}} \lim_{\eta \to 0^+} \int_{-\infty}^{+\infty} \frac{e^{-\mathrm{i}\, \omega t}}{\omega + \mathrm{i}\, \eta} \dd{\omega} 
\end{equation}
应用 Sokhotski–Plemelj 定理, 我们有:
\begin{equation}
\frac{1}{\omega + \mathrm{i}\, \eta} = \mathcal{P} \frac{1}{\omega} - \mathrm{i}\, \pi \delta(\omega)
\end{equation}
因此, 我们有:
\begin{equation}
\Theta(t) = -\frac{1}{2\pi \mathrm{i}} \left[ \mathcal{P} \int_{-\infty}^{+\infty} \frac{e^{-\mathrm{i}\, \omega t}}{\omega} \dd{\omega} - \mathrm{i}\, \pi \int_{-\infty}^{+\infty} e^{-\mathrm{i}\, \omega t} \delta(\omega) \dd{\omega} \right]
\end{equation}
我们计算第二部分:
\begin{equation}
-\frac{1}{2\pi \mathrm{i}} \left( - \mathrm{i}\, \pi \int_{-\infty}^{+\infty} e^{-\mathrm{i}\, \omega t} \delta(\omega) \dd{\omega} \right) = \frac{1}{2}
\end{equation}
第一部分是个有趣的函数, 我们不多说了:
\begin{equation}
-\frac{1}{2\pi \mathrm{i}} \mathcal{P} \int_{-\infty}^{+\infty} \frac{e^{-\mathrm{i}\, \omega t}}{\omega} \dd{\omega} = \frac{1}{2} \mathrm{sgn}(t)
\end{equation}
其中 $\mathrm{sgn}(t)$ 是符号函数, 当 $t>0$ 时为 $+1$, 当 $t<0$ 时为 $-1$.
所以最后我们得到了:
\begin{equation}
\Theta(t) = \frac{1}{2} \left[ \mathrm{sgn}(t) + 1 \right]
\end{equation}
这正是 $\Theta$ 函数的定义.
\end{mexample}


有了这个恒等式, 我们就可以把谱密度和格林函数联系起来.
上一个小节, 我们说了, 对于不含时的 Hamiltonian, 我们可以写出它的频域推迟格林函数为:
\begin{equation}
\tilde{G}^R(E) = \lim_{\eta \to 0^+} \frac{1}{E - \hat{H} + \mathrm{i}\, \eta}
\end{equation}
对他进行谱分解, 考虑本征谱:
\begin{equation}
\hat{H} \ket{n} = E_n \ket{n}
\end{equation}
那么我们就有:
\begin{equation}
\tilde{G}^R(E) = \sum_n \lim_{\eta \to 0^+} \frac{\ketbra{n}}{E - E_n + \mathrm{i}\, \eta} 
\end{equation}
应用 Sokhotski–Plemelj 定理, 我们有:
\begin{equation}
\tilde{G}^R(E) = \sum_n \left[ \mathcal{P} \frac{1}{E - E_n} - \mathrm{i}\,\pi \delta(E - E_n) \right] \ketbra{n}
\end{equation}
这个时候天然出现了谱密度这个东西, 就是虚部:
\begin{equation}
\Im \tilde{G}^R(E) = -\pi \sum_n \delta(E - E_n) \ketbra{n} = -\pi \hat{\rho}(E)
\end{equation}
如果我们求他的期望值:
\begin{equation}
\Im \mel{\psi}{\tilde{G}^R(E)}{\psi} = -\pi \sum_n |\braket{n}{\psi}|^2 \delta(E - E_n) = -\pi S_{\ket{\psi}}(E)
\end{equation}
从而我们得到:
\begin{equation}
S_{\ket{\psi}}(E) = -\frac{1}{\pi} \Im \mel{\psi}{\tilde{G}^R(E)}{\psi}
\end{equation}


说来说去, 谱密度到底有什么用呢?
\reminder{注意啊, 我们目前的讨论都是基于不含时的 Hamiltonian 的.}
首先, 他解决的第一个问题很简单, 就是告诉我们, 一个粒子, 一开始处于某个态 $\ket{\psi}$, 那么过了一段时间, 他保持在这个状态的概率有多大.
有的文献管这个叫做\newterm{生存振幅, survival amplitude}:
\begin{equation}
A_{\psi}(t) = \mel{\psi}{\mathrm{e}^{-\mathrm{i}\, \hat{H} t / \hbar}}{\psi}
\end{equation}
考虑谱分解:
\begin{equation}
A_{\psi}(t) = \sum_{n,m} \braket{\psi}{n} \mel{n}{\mathrm{e}^{-\mathrm{i}\, \hat{H} t / \hbar}}{m} \braket{m}{\psi} = \sum_n |\braket{n}{\psi}|^2 \mathrm{e}^{-\mathrm{i}\, E_n t / \hbar}
\end{equation}
如果我们再插入一个恒等式:
\begin{equation}
\mathbb{1} = \int_{-\infty}^{+\infty} \delta(E - \hat{H}) \dd{E}
\end{equation}
我们有:
\begin{equation}
A_{\psi}(t) = \int_{-\infty}^{+\infty} \sum_n |\braket{n}{\psi}|^2 \delta(E - E_n) \mathrm{e}^{-\mathrm{i}\, E t / \hbar} \dd{E}
\end{equation}
注意到谱密度的定义, 我们有:
\begin{equation}
A_{\psi}(t) = \int_{-\infty}^{+\infty} S_{\ket{\psi}}(E) \mathrm{e}^{-\mathrm{i}\, E t / \hbar} \dd{E}
\end{equation}
也就是说, 生存振幅就是谱密度的傅里叶变换.
这就很有意思了, 因为谱密度告诉我们, 态 $\ket{\psi}$ 在不同能量处的“分量”有多大, 而生存振幅告诉我们, 态 $\ket{\psi}$ 随时间的演化情况.
所以, 态 $\ket{\psi}$ 在不同能量处的“分量”有多大, 决定了它随时间的演化情况.
这就是谱密度的第一个用途.

\begin{mexample}{单色波}
一个自然而然的推论: 如果态 $\ket{\psi}$ 恰好是某个本征态 $\ket{n}$, 那么他的谱密度就是:
\begin{equation}
S_{\ket{n}}(E) = \delta(E - E_n)
\end{equation}
那么他的生存振幅就是:
\begin{equation}
A_{n}(t) = \int_{-\infty}^{+\infty} \delta(E - E_n) \mathrm{e}^{-\mathrm{i}\, E t / \hbar} \dd{E} = \mathrm{e}^{-\mathrm{i}\, E_n t / \hbar}
\end{equation}
也就是说, 态 $\ket{n}$ 会一直保持不变, 只是多了一个相位因子.
这就是量子力学中的单色波 (monochromatic wave).
\end{mexample}

\begin{mexample}{两个本征态的叠加}
如果态 $\ket{\psi}$ 是两个本征态的叠加, 比如:
\begin{equation}
\ket{\psi} = \frac{1}{\sqrt{2}} (\ket{n_1} + \ket{n_2})
\end{equation}
那么他的谱密度就是:
\begin{equation}
S_{\ket{\psi}}(E) = \frac{1}{2} \left[ \delta(E - E_{n_1}) + \delta(E - E_{n_2}) \right]
\end{equation}
那么他的生存振幅就是:
\begin{equation}
A_{\psi}(t) = \frac{1}{2} \left[ \mathrm{e}^{-\mathrm{i}\, E_{n_1} t / \hbar} + \mathrm{e}^{-\mathrm{i}\, E_{n_2} t / \hbar} \right]
\end{equation}
概率为:
\begin{equation}
|A_{\psi}(t)|^2 = \frac{1}{2} \cos^2 \left( \frac{E_{n_2} - E_{n_1}}{2 \hbar} t \right)
\end{equation}
也就是说, 态 $\ket{\psi}$ 会在两个本征态之间振荡.
\end{mexample}

\begin{mexample}{有限宽度的谱}
如果态 $\ket{\psi}$ 的谱密度在某个能量范围内是连续的, 比如:
\begin{equation}
S_{\ket{\psi}}(E) = \frac{1}{\pi} \frac{\Gamma / 2}{(E - E_0)^2 + (\Gamma / 2)^2}
\end{equation}
这是一个 Lorentzian 分布, 宽度为 $\Gamma$, 中心在 $E_0$.
那么他的生存振幅就是:
\begin{equation}
A_{\psi}(t) = \int_{-\infty}^{+\infty} \frac{1}{\pi} \frac{\Gamma / 2}{(E - E_0)^2 + (\Gamma / 2)^2} \mathrm{e}^{-\mathrm{i}\, E t / \hbar} \dd{E}
\end{equation}
这个积分可以通过留数定理计算, 结果是:
\begin{equation}
A_{\psi}(t) = \mathrm{e}^{-\mathrm{i}\, E_0 t / \hbar} \mathrm{e}^{-\Gamma t / (2 \hbar)}
\end{equation}
计算概率为:
\begin{equation}
|A_{\psi}(t)|^2 = \mathrm{e}^{-\Gamma t / \hbar}
\end{equation}
也就是说, 态 $\ket{\psi}$ 会以指数形式衰减, 他会随着时间的推移逐渐转化成其他态.
这个问题我们会再次强调的
\end{mexample}
总而言之, 谱密度越窄, 态的时间演化就越接近单色波; 谱密度越宽, 态的时间演化就越复杂, 可能会出现衰减等现象.


接下来, 我们介绍和谱密度息息相关的一个概念: \newterm{态密度, density of states}.
我们之前说了, 谱密度告诉我们, 对于一个态 $\ket{\psi}$, 它在不同能量处的分量有多大.
这里面就有一个观察: 对于 $\ket{\psi}$ 来说, 长的和他自己越像 (交叠积分越大) 的本征态, 他在那个能量处的分量就越大.
所以谱密度还是在针对某个特定态 $\ket{\psi}$ 来说的.
那么如果我们不针对某个特定态, 而是想知道, 在某个能量处, 有多少个本征态呢?
这就引出了态密度的概念.
\begin{mdefinition}{态密度, density of states}
设 $\hat{H}$ 的本征值为 $E_n$, 本征态为 $\ket{n}$.
我们考虑一组完备的正交归一化基底 $\{\ket{\phi_i}\}$, 对于其中一个基底态 $\ket{\phi_i}$, 我们定义他在能量 $E$ 处谱密度为:
\begin{equation}
S_{\ket{\phi_i}}(E) = \mel{\phi_i}{\hat{\rho}(E)}{\phi_i} = \sum_n |\braket{n}{\phi_i}|^2 \delta(E - E_n)
\end{equation}
态密度就是这组基底态在能量 $E$ 处的谱密度的总和:
\begin{equation}
N(E) = \sum_i S_{\ket{\phi_i}}(E) = \sum_{i,n} |\braket{n}{\phi_i}|^2 \delta(E - E_n)
\end{equation}
\end{mdefinition}
这一下就告诉我们, $S_{\ket{\phi_i}}(E)$ 是依赖于 $\ket{\phi_i}$ 的, 而态密度 $N(E)$ 则是和基底无关的.
实际上我们还能化简上述推导:
\begin{equation}
N(E) = \sum_{i,n} |\braket{n}{\phi_i}|^2 \delta(E - E_n) = \sum_n \left( \sum_i |\braket{n}{\phi_i}|^2 \right) \delta(E - E_n) = \sum_n \delta(E - E_n)
\end{equation}
也就是说, 我们把所有的满足能量为 $E$ 的本征态 $\ket{n}$ 都数了一遍, 这就是态密度.
这里面存在一个可能的疑惑: \question{难道不是一个对应每个 $E$ 都只有一个 $E_n$}.
这个答案就很简单了: 不是的, 很多体系的 Hamiltonian 都存在简并现象, 也就是说, 可能有多个本征态对应同一个本征值, 而且 $n$ 是用来编号本征态的, 不是编号能量的, 所以很有可能:
\begin{equation}
E_{n_1} = E_{n_2} = \cdots = E_{n_k} = E
\end{equation}
就是说, 态 $\ket{1}$, $\ket{2}$, ..., $\ket{k}$ 都对应同一个能量 $E$.
所以态密度就是告诉我们, 在某个能量 $E$ 处, 有多少个本征态.
从这个角度上说, 谱密度是针对某个态的, 而态密度则是针对整个体系的.


现在我们把所有的内容串一下, 首先我们注意到, 频域推迟格林函数为 (算符形式, 我们忽略了那个$\tilde{}$, 然后标上了算符标志):
\begin{equation}
\hat{G}^R(E) = \lim_{\eta \to 0^+} \frac{1}{E - \hat{H} + \mathrm{i}\, \eta}
\end{equation}
从而我们的谱密度 (spectral density operator) 就是:
\begin{equation}
\hat{\rho}(E) = -\frac{1}{\pi} \Im \hat{G}^R(E)
\end{equation}
态 $\ket{\psi}$ 的谱密度就是:
\begin{equation}
S_{\ket{\psi}}(E) = -\frac{1}{\pi} \Im \mel{\psi}{\hat{G}^R(E)}{\psi}
\end{equation}
态密度就是:
\begin{equation}
N(E) = -\frac{1}{\pi} \Im \Tr \hat{G}^R(E)
\end{equation}
这里面的 $\Tr$ 是算符的迹运算, 定义为:
\begin{equation}
\Tr \hat{A} = \sum_i \mel{\phi_i}{\hat{A}}{\phi_i}
\end{equation}
其中 $\{\ket{\phi_i}\}$ 是一组完备的正交归一化基底.


最后, 我们再讲一个非常有用的例子.
我们之前定义了任意态的谱密度为:
\begin{equation}
S_{\ket{\psi}}(E) = \mel{\psi}{\hat{\rho}(E)}{\psi} = \mel{\psi}{\delta(E - \hat{H})}{\psi}
\end{equation}
我们自然是可以选择位置本征态 $\ket{\bm{x}}$ 作为态 $\ket{\psi}$ 的, 就相当于, 我们把一个探针放到空间中的某个位置 $\bm{x}$, 然后测量这个位置处的谱密度.
这就是所谓的\newterm{局域态密度, local density of states}:
\begin{equation}
\rho(\bm{x}, E) = S_{\ket{\bm{x}}}(E) = \mel{\bm{x}}{\delta(E - \hat{H})}{\bm{x}}
\end{equation}
利用本征谱展开, 我们有:
\begin{equation}
\rho(\bm{x}, E) = \sum_n |\braket{n}{\bm{x}}|^2 \delta(E - E_n) = \sum_n |\psi_n(\bm{x})|^2 \delta(E - E_n)
\end{equation}
其中 $\psi_n(\bm{x}) = \braket{\bm{x}}{n}$ 是本征态 $\ket{n}$ 在位置空间的波函数表示.
局域态密度告诉我们, 在位置 $\bm{x}$ 处, 能量为 $E$ 时候, 我们找到例子的概率密度有多大.
\reminder{注意啊, 局域态密度和态密度是不同的概念, 态密度是针对整个体系的, 而局域态密度是针对空间中的某个位置的.}
同样的, 我们也可以把局域态密度和格林函数联系起来:
\begin{equation}
\rho(\bm{x}, E) = -\frac{1}{\pi} \Im \mel{\bm{x}}{\hat{G}^R(\bm{x}, \bm{x}; E)}{\bm{x}}
\end{equation}
这就是说, 如果我们知道了格林函数在位置空间的表示 (对角元), 那么我们就可以计算出局域态密度.

这个理论在凝聚态物理中有非常重要的应用, 比如扫描隧道显微镜 (STM) 的工作原理, 就是基于局域态密度的测量.
我们有一个尖端, 把它放到样品表面附近, 然后施加一个小的电压.
电子会从尖端隧道到样品表面, 隧道电流的大小, 就和样品表面的局域态密度成正比.
通过扫描尖端的位置, 我们就可以得到样品表面的局域态密度分布, 从而得到样品的表面结构信息.
他可以直接固定电流, 然后扫描 $\bm{x}$, 记录下不同位置处的电压, 直接反映出电子云的整体形状.
他还可以固定位置, 然后扫描电压, 记录下不同电压下的电流, 直接反映出局域态密度随能量的分布情况, 这就是 scanning tunneling spectroscopy (STS).
这样一旦看到了格林函数的虚部, 我们就能直接得到局域态密度, 进而得到很多有用的信息.